\documentclass[conference]{IEEEtran}
\IEEEoverridecommandlockouts
% The preceding line is only needed to identify funding in the first footnote. If that is unneeded, please comment it out.
\usepackage{cite}
\usepackage{amsmath,amssymb,amsfonts}
\usepackage{algorithmic}
\usepackage{graphicx}
\usepackage{textcomp}
\usepackage{xcolor}
\usepackage{hyperref}
\usepackage{seqsplit}
\def\BibTeX{{\rm B\kern-.05em{\sc i\kern-.025em b}\kern-.08em
    T\kern-.1667em\lower.7ex\hbox{E}\kern-.125emX}}
\begin{document}

\title{CPEN 442 Project Proposal}

\author{\IEEEauthorblockN{Mike Yue, Student Number: 24583156}}
\author{\IEEEauthorblockN{Ben Henaghan, Student Number: 96671466}}
\author{\IEEEauthorblockN{Austine Yapp Student Number: TODO}}
\author{\IEEEauthorblockN{Scott Wang, Student Number: TOOO}}

\maketitle

\textbf{\textcolor{red}{REMEMBER TO WRITE IN PAST TENSE}}

\begin{abstract}
	This project addressed the issue of keypad locks being highly vulnerable to a variety of simple attacks by implementing an internet conected keypad which utilised one-time passcodes. This retained the usefulness of a keypad (as opposed to a key lock) while greatly increasing security.
\end{abstract}

\section{Introduction}

\section{Current Solutions}

\section{Our Implementation}

\section{Comparison to Competitng Solutions}

\section{Plan for Project}

\end{document}
