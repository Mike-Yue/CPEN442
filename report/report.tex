\documentclass[conference]{IEEEtran}
\IEEEoverridecommandlockouts
% The preceding line is only needed to identify funding in the first footnote. If that is unneeded, please comment it out.
\usepackage{cite}
\usepackage{amsmath,amssymb,amsfonts}
\usepackage{algorithmic}
\usepackage{graphicx}
\usepackage{textcomp}
\usepackage{xcolor}
\def\BibTeX{{\rm B\kern-.05em{\sc i\kern-.025em b}\kern-.08em
    T\kern-.1667em\lower.7ex\hbox{E}\kern-.125emX}}
\begin{document}

\title{Keypad Entry System\\
Group 4: Gamechangers
}

\author{\IEEEauthorblockN{1\textsuperscript{st} Ben Henaghan}
\and
\IEEEauthorblockN{2\textsuperscript{nd} Scott Wang}
\and
\IEEEauthorblockN{3\textsuperscript{rd} Austine Yapp}
\and
\IEEEauthorblockN{4\textsuperscript{th} Mike Yue}
}

\maketitle

\begin{abstract}
This document is a model and instructions for \LaTeX.
This and the IEEEtran.cls file define the components of your paper [title, text, heads, etc.]. *CRITICAL: Do Not Use Symbols, Special Characters, Footnotes, 
or Math in Paper Title or Abstract.
\end{abstract}

\begin{IEEEkeywords}
component, formatting, style, styling, insert
\end{IEEEkeywords}

\section{Introduction}
\subsection{Problems with traditional keypad entry systems}

\subsection{Proposed Solution}
	We designed a digital-lock system that employs the use of dynamically generated TOTP access codes instead of conventional static codes. Our system uses fingerprint biometrics for authentication as an alternative to logging in using a username and password. Fingerprint authentication has been proven to be more usable and efficient than entering a username and password \cite{b2}, minimizing the amount of effort and time required on the user’s part. Once the user is authenticated, they can then generate a TOTP for unlocking a particular digital door-lock. Upon unlocking the digital door-lock, the valid TOTP is immediately destroyed and rendered invalid. 
	There has been similar work done by CMU on the Grey system \cite{b3}, a device-enabled authorization system utilizing smartphones to unify access controls. However, the system still utilizes a static PIN code in order to activate a user’s private key for authentication. This leaves the system still highly vulnerable to adversarial attacks such as shoulder surfing. Our proposed system thus hopes to remove the need for static PIN codes through TOTP dynamic PIN codes. 

\subsection{Contributions}
The work on the prototype was split naturally over the three components --- The mobile application, server application and code which would run on the actual lock hardware.
It was deemed that the android (mobile) application would form the largest amount of work, so that was split between Ben Henaghan and Scott Wang who both had some experience writing mobile applications.
Mike Yue developed the server code and Austine Yapp produced the software for the lock hardware.

\section{Related Work}
\subsection{Access Control System}
	Access control systems are classified by an electrical control and mechanical door-lock analog according to their operating method. It is possible to classify electronic door-locks based on their recognition technology, namely card recognition, number input method, and methods of bio-information recognition. These categories mainly fall into the three universally recognized authentication factors that exist today: what you know (passwords), what you have (tokens, keys) and what you are (biometrics). 
Passwords are considered to be one of the easiest targets for hackers. Many companies are therefore searching for more secure methods to protect the information of their customers and employees. On the other hand, biometrics are known to be highly secure and used in specific organizations. However, the hardware and maintenance cost required to upkeep such a system is relatively expensive. One way of bypassing the hardware requirements is to utilize the existing resources on mobile phones such as fingerprint scanners and/or facial recognition technology.

<<<<<<< Updated upstream
\section{Adversary Model}
=======
\subsection{Similar Systems}
	One Time Password (OTP) is a single-use password that is valid for only one login session or transaction. It is often used in various fields such as banking to provide an additional layer of security. OTP challenge-response methods can be broadly classified into event-synchronization, time-synchronization, and a combination of methods \cite{b4} \cite{b5}. Every challenge-response OTP method requires users to provide a response to a challenge. The new password is traditionally generated using a mathematical algorithm based on a challenge, which is eventually used to authenticate the users. Event-synchronization method relies on the HMAC-based One-Time Password (HOTP) algorithm published in IETF RFC 4226 \cite{b6} to generate a new password. Time-synchronization OTP method is an extension of HOTP published in IETF RFC 6238 \cite{b7} which uses the current time instead of mathematical algorithms in order to generate a new password that remains valid for the duration of a specified time step. While both OTP methods offer single-use passwords, HOTP remains valid until it is used, or until a subsequent OTP is generated. In contrast, TOTP only has one valid OTP at any given time, with a smaller predefined validation window. HOTPs are thus more susceptible to a brute-force attack due to the larger validation window.
	One similar system is the Grey system at CMU which proposes the utilization of “smartphones” for unifying access control to both physical (e.g. doors, safes) and virtual (e.g. files, accounts) resources. The Grey system suggests the use of smartphones as the central agent through which access control is managed, namely because of their prevalent adoption in society as well as the hardware capabilities present that would enable applications to take full advantage of the rich computation, communication, and interface capabilities. It employs proof-carrying authorization (PCA) \cite{b8} extended with a new distributed proving technique with considerable efficiency improvements \cite{b9}. Our designs are similar in the fact that both our systems utilize smartphones for client-side authentication. Also, both designs are intended to conveniently authorize access to other people. 
	One difference in our design from the Grey system is the initial set up of the lock system. In the Grey system, this initial set up is done via either Bluetooth discovery or using the phone’s camera to photograph a two-dimensional barcode. Our system instead requires the user to activate the lock via a unique serial number upon first use. In order to do so, the user has to first be logged in to an account associated with our server. Subsequent attempts to unlock the digital lock will require either user account log-in or biometric authorization. Furthermore, our system employs the use of TOTP as the challenge-response OTP mechanism. Users of the Grey system utilize static PIN codes in order to activate their individual private key for authorization. Even though the PIN code entry system is abstracted to the smartphone, it is still highly vulnerable to the same adversarial attacks such as shoulder surfing and/or social engineered attacks.

>>>>>>> Stashed changes

\section{Adversary Model}

<<<<<<< Updated upstream
\section{System Prototype}
=======
\section{System Design}
\subsection{Overview}
	Instead of a static pin code, we use a dynamically-generated TOTP access code. First, the user must authenticate themselves using their mobile device’s fingerprint scanner to unlock the device’s unique fingerprint secret which was generated when the application was first started. Next, the application sends the fingerprint secret to the server for authentication. The server then sends the list of authorized locks for the user to unlock. Upon selecting a particular lock, the server generates a valid TOTP code associated with that lock, and this code is sent to the user through the mobile app. TOTP is an extension of the HOTP algorithm that uses a secret shared key and the current timestamp as inputs to a cryptographic HMAC hash function. User input in the lock is then sent to the server and verified on the server-side. If the code entered is valid, the server returns a success message and unlocks the lock. Otherwise, the server returns a forbidden message and the lock remains unlocked.
	Due to the design of the system, a valid code will only be generated and associated with the specific lock on the server database upon request by an authorized user. Without an authorized user’s request to the server, no valid code will be associated with the lock. Any code that is entered will thus be denied entry. This essentially eliminates the threat of brute-force attempts by adversaries who attempt to guess the lock’s code. 
Moreover, adversaries may attempt to gain access to the access codes through a shoulder surfing attack. The system employs the use of expiring TOTP codes that are valid only for a single-use. After which, the server immediately deletes the associated valid code. This means that even if an adversary were to successfully identify the user’s OTP code, it becomes useless upon use. This effectively nullifies the threat of adversarial replay attacks.
	Furthermore, an adversary may attempt to gain access to the authorized user’s device through a socially engineered attack. Even if the attacker were to have unrestricted access to the user’s phone, they would not be able to generate a valid TOTP without first authenticating themselves with their fingerprint biometrics. Communication with the server will only be established if the user’s identity is authenticated. This way, any requests made to the server to generate a TOTP can only be done by an authorized user.
	Finally, the main goal for our design is to provide significantly greater security of traditional digital locks while minimizing the additional inconvenience imposed on users. In order to do so, we kept our system compatible with existing digital lock systems and as small as possible in order to allow for retrofitting of our proposed design. Though more effort and time will be required on the part of the user and their device, we kept this as seamless and effortless as possible. 


\section{System Prototype}
\subsection{Hardware}
	We used a RaspberryPi Zero W to build the prototype of our digital lock system. We connected a numeric keypad to simulate a traditional keypad door lock to provide input to the RaspberryPi. Upon entry by a user, the lock system uses the requests Python module to establish a connection with the server via HTTPS. The lock system then retrieves its own serial number and sends it along with the code entered by the user. On the server-side, the code is checked with the lock associated with the serial number. If the code is valid, the server returns a success message. Otherwise, a forbidden message is returned. The digital lock then checks the response from the server and unlocks the lock if and only if the server managed to authenticate the user’s code. On the digital lock side, communication with the server is not authenticated. This is intentional in order to allow a third-party to gain access through an authorized user. For example, a domestic cleaner (unauthorized) may be granted one-time access by a homeowner (authorized).
>>>>>>> Stashed changes

\section{Evaluation}

\section{Discussion}

\section{Conclusion}

\begin{table}[htbp]
\caption{Table Type Styles}
\begin{center}
\begin{tabular}{|c|c|c|c|}
\hline
\textbf{Table}&\multicolumn{3}{|c|}{\textbf{Table Column Head}} \\
\cline{2-4} 
\textbf{Head} & \textbf{\textit{Table column subhead}}& \textbf{\textit{Subhead}}& \textbf{\textit{Subhead}} \\
\hline
copy& More table copy$^{\mathrm{a}}$& &  \\
\hline
\multicolumn{4}{l}{$^{\mathrm{a}}$Sample of a Table footnote.}
\end{tabular}
\label{tab1}
\end{center}
\end{table}

% \begin{figure}[htbp]
% \centerline{\includegraphics{fig1.png}}
% \caption{Example of a figure caption.}
% \label{fig}
% \end{figure}

Figure Labels: Use 8 point Times New Roman for Figure labels. Use words 
rather than symbols or abbreviations when writing Figure axis labels to 
avoid confusing the reader. As an example, write the quantity 
``Magnetization'', or ``Magnetization, M'', not just ``M''. If including 
units in the label, present them within parentheses. Do not label axes only 
with units. In the example, write ``Magnetization (A/m)'' or ``Magnetization 
\{A[m(1)]\}'', not just ``A/m''. Do not label axes with a ratio of 
quantities and units. For example, write ``Temperature (K)'', not 
``Temperature/K''.

\section*{Acknowledgment}

The preferred spelling of the word ``acknowledgment'' in America is without 
an ``e'' after the ``g''. Avoid the stilted expression ``one of us (R. B. 
G.) thanks $\ldots$''. Instead, try ``R. B. G. thanks$\ldots$''. Put sponsor 
acknowledgments in the unnumbered footnote on the first page.

\section*{References}

Please number citations consecutively within brackets \cite{b1}. The 
sentence punctuation follows the bracket \cite{b2}. Refer simply to the reference 
number, as in \cite{b3}---do not use ``Ref. \cite{b3}'' or ``reference \cite{b3}'' except at 
the beginning of a sentence: ``Reference \cite{b3} was the first $\ldots$''

Number footnotes separately in superscripts. Place the actual footnote at 
the bottom of the column in which it was cited. Do not put footnotes in the 
abstract or reference list. Use letters for table footnotes.

Unless there are six authors or more give all authors' names; do not use 
``et al.''. Papers that have not been published, even if they have been 
submitted for publication, should be cited as ``unpublished'' \cite{b4}. Papers 
that have been accepted for publication should be cited as ``in press'' \cite{b5}. 
Capitalize only the first word in a paper title, except for proper nouns and 
element symbols.

For papers published in translation journals, please give the English 
citation first, followed by the original foreign-language citation \cite{b6}.

\begin{thebibliography}{00}
\bibitem{b1} G. Eason, B. Noble, and I. N. Sneddon, ``On certain integrals of Lipschitz-Hankel type involving products of Bessel functions,'' Phil. Trans. Roy. Soc. London, vol. A247, pp. 529--551, April 1955.
<<<<<<< Updated upstream
\bibitem{b2} J. Clerk Maxwell, A Treatise on Electricity and Magnetism, 3rd ed., vol. 2. Oxford: Clarendon, 1892, pp.68--73.
\bibitem{b3} I. S. Jacobs and C. P. Bean, ``Fine particles, thin films and exchange anisotropy,'' in Magnetism, vol. III, G. T. Rado and H. Suhl, Eds. New York: Academic, 1963, pp. 271--350.
\bibitem{b4} K. Elissa, ``Title of paper if known,'' unpublished.
\bibitem{b5} R. Nicole, ``Title of paper with only first word capitalized,'' J. Name Stand. Abbrev., in press.
\bibitem{b6} Y. Yorozu, M. Hirano, K. Oka, and Y. Tagawa, ``Electron spectroscopy studies on magneto-optical media and plastic substrate interface,'' IEEE Transl. J. Magn. Japan, vol. 2, pp. 740--741, August 1987 [Digests 9th Annual Conf. Magnetics Japan, p. 301, 1982].
\bibitem{b7} M. Young, The Technical Writer's Handbook. Mill Valley, CA: University Science, 1989.
=======
\bibitem{b2} G D Lam, “Evaluating the Usability of an Apple Touch ID-Based Access Control System” University of British Columbia, April 2015.
\bibitem{b3} L. Bauer, S. Garriss, J. M. Mccune, M. K. Reiter, J. Rouse, and P. Rutenbar, “Device-Enabled Authorization in the Grey System,” Jan. 2005.
\bibitem{b4} D. Choi, W. Kim and D. Won, “One-Time Password Technology Analysis and Standardization Trend,” Review of KIISC, vol. 17,no. 3, pp. 12-17, Jun. 2007.
\bibitem{b5} S. Seo and W. Kang, “OTP Technology Status and OTP Introduction Example,” Review of KIISC, vol. 17, no. 3, pp. 18-25, Jun. 2007.
\bibitem{b6} ”RFC 4226 - HOTP: An HMAC-Based One-Time Password Algorithm”, Tools.ietf.org, 2016. [Online]. Available: https://tools.ietf.org/html/rfc4226. [Accessed: 05- Dec- 2019].
\bibitem{b7} ”RFC 6238 - TOTP: Time-based One-time Password Algorithm”, Tools.ietf.org, 2016. [Online]. Available: https://tools.ietf.org/html/rfc6238. [Accessed: 05- Dec- 2019].
\bibitem{b8} A. W. Appel and E. W. Felten. Proof-carrying authentication. In Proc. 6th ACM Conference on Computer and Communications Security, Nov. 1999.
\bibitem{b9} L. Bauer, S. Garriss, and M. K. Reiter. Distributed proving in access-control systems. In Proc. 2005 IEEE Symposium on Security and Privacy, May 2005.
>>>>>>> Stashed changes
\end{thebibliography}
\vspace{12pt}
\color{red}
IEEE conference templates contain guidance text for composing and formatting conference papers. Please ensure that all template text is removed from your conference paper prior to submission to the conference. Failure to remove the template text from your paper may result in your paper not being published.

\end{document}
