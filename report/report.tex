\documentclass[conference]{IEEEtran}
\IEEEoverridecommandlockouts
% The preceding line is only needed to identify funding in the first footnote. If that is unneeded, please comment it out.
\usepackage{cite}
\usepackage{amsmath,amssymb,amsfonts}
\usepackage{algorithmic}
\usepackage{graphicx}
\usepackage{textcomp}
\usepackage{xcolor}
\def\BibTeX{{\rm B\kern-.05em{\sc i\kern-.025em b}\kern-.08em
    T\kern-.1667em\lower.7ex\hbox{E}\kern-.125emX}}
\begin{document}

\title{Keypad Entry System\\
Group 4: Gamechangers
}

\author{\IEEEauthorblockN{1\textsuperscript{st} Ben Henaghan}
\and
\IEEEauthorblockN{2\textsuperscript{nd} Scott Wang}
\and
\IEEEauthorblockN{3\textsuperscript{rd} Austine Yapp}
\and
\IEEEauthorblockN{4\textsuperscript{th} Mike Yue}
}

\maketitle

\begin{abstract}
This document is a model and instructions for \LaTeX.
This and the IEEEtran.cls file define the components of your paper [title, text, heads, etc.]. *CRITICAL: Do Not Use Symbols, Special Characters, Footnotes, 
or Math in Paper Title or Abstract.
\end{abstract}

\begin{IEEEkeywords}
component, formatting, style, styling, insert
\end{IEEEkeywords}

\section{Introduction}
\subsection{Problems with traditional keypad entry systems}

\subsection{Proposed Solution}

\subsection{Contributions}
The work on the prototype was split naturally over the three components --- The mobile application, server application and code which would run on the actual lock hardware.
It was deemed that the android (mobile) application would form the largest amount of work, so that was split between Ben Henaghan and Scott Wang who both had some experience writing mobile applications.
Mike Yue developed the server code and Austine Yapp produced the software for the lock hardware.

\section{Related Work}

\section{Adversary Model}

\section{System Design}

\section{System Prototype}

\section{Evaluation}
\newpage
\section{Discussion}
\textbf{Discuss pros and cons of your design, as well as limitations and advantages of it. This discussion should integrate the related work, the motivation for the design, the adversary model, the ideas of the design, and the results of the evaluation. This section should be about one page.}
\newline

In this section, we will discuss the advantages and benfits of our system prototype and its broader design.
We will then examine its limitations and disadvantages.
Finally, we will consider the findings of our project overall, with these aforementioned `pros' and `cons' in mind.

Our prototype system is complete enough for us to draw useful conclusions about a similar potential product but is by no means a complete system in of itself.
The simple design allowed the development team to produce an MVP (Minimum Viable Product) and then iterate on this initial product to build our prototype system.
Fundamentally, our system only comprises a subset of the hardware needed for an electronic door lock, namely the authentication and control electronics.
Many non-core processes have been omitted from our prototype, namely a secure lock enrollment method and a full logging system.
Although our prototype was missing these features, we were still able to collect significant data about the usability of this kind of system, which could easily be incorporated into a future commercial security device.

\subsection{Advantages}
User testing showed that users appreciated the clear security benefits of our system but did find it to be a slight inconvenience.
The trade-off of security and convenience is a core problem when designing security products for use by the general public; users will be hesitant to opt-in to enhanced security measures if they deem it to be a significant inconvenience.
It has been shown that almost two-thirds of users will select the least secure and most convenienent security option if given the choice \cite{bben1}.
Our system seeked to strike a good balance between the extra security benefits of TOTP codes and the convenience of a numeric keypad lock.

All types of shoulder-surfing and replay attacks are completely defeated by our device, meaning that there aren't many possible `subtle' (ie non-destructive) attacks against our system.
However, it is worth noting that the Wi-Fi connected electronics, mobile application and server are all attack surfaces not present in standard keypad-operated access control systems.
Overall risk incurred by the device is therefore reduced, as the vulnerability value is significantly decreased.

Given our adversay model, our device successfully resists most known capabilities, such as replay attacks with pinhole cameras.
The adversary model does also assume that they would be able to gain access to the user's phone using a social engineering attack, and given access to the phone it is possible that the adverary would be able to comprimise the system.
This is because if the threat agent is able to unlock the phone, they should be able to enrol their own biometrics and therefore activate the code generation in the android application.
The risk of access in this way is mitigated in a few different ways.
Enrolling a fingerprint with android  and then generating an unlock code would typically takes upwards of a minute, which would likely arrouse the suspiscion of the owner.
The impact of this kind of attack could be further reduced if guidance was added to the android application telling users to be aware of such social engineering attacks and not to share their phone wiht strangers without being aware of what they're doing.
The other mitigation of this kind of attack relies on the expiry time of the single-use code, as the adversary would have to get to the door lock in time to use the comprimised code --- which is likely take a significant amount of time due phsycial distance.
A sophisticated attack could have another adversary located closer to the door, but this level of organization is uncharacteristic for an oppourtunistic threat agent.

\subsection{Limitations}

As mentioned before, the phsycial security of our prototype device was out of scope for this project, but would be an important part of the design if we wanted to bring a product like this to market.
Our prototype, however, did suffer from a few limitations which impeded both security and usability.



\subsection{Overall Findings}



\section{Conclusion}

% \begin{table}[htbp]
% \caption{Table Type Styles}
% \begin{center}
% \begin{tabular}{|c|c|c|c|}
% \hline
% \textbf{Table}&\multicolumn{3}{|c|}{\textbf{Table Column Head}} \\
% \cline{2-4} 
% \textbf{Head} & \textbf{\textit{Table column subhead}}& \textbf{\textit{Subhead}}& \textbf{\textit{Subhead}} \\
% \hline
% copy& More table copy$^{\mathrm{a}}$& &  \\
% \hline
% \multicolumn{4}{l}{$^{\mathrm{a}}$Sample of a Table footnote.}
% \end{tabular}
% \label{tab1}
% \end{center}
% \end{table}

% \begin{figure}[htbp]
% \centerline{\includegraphics{fig1.png}}
% \caption{Example of a figure caption.}
% \label{fig}
% \end{figure}


% \section*{References}

% Please number citations consecutively within brackets \cite{b1}. The 
% sentence punctuation follows the bracket \cite{b2}. Refer simply to the reference 
% number, as in \cite{b3}---do not use ``Ref. \cite{b3}'' or ``reference \cite{b3}'' except at 
% the beginning of a sentence: ``Reference \cite{b3} was the first $\ldots$''

% Number footnotes separately in superscripts. Place the actual footnote at 
% the bottom of the column in which it was cited. Do not put footnotes in the 
% abstract or reference list. Use letters for table footnotes.

% Unless there are six authors or more give all authors' names; do not use 
% ``et al.''. Papers that have not been published, even if they have been 
% submitted for publication, should be cited as ``unpublished'' \cite{b4}. Papers 
% that have been accepted for publication should be cited as ``in press'' \cite{b5}. 
% Capitalize only the first word in a paper title, except for proper nouns and 
% element symbols.

% For papers published in translation journals, please give the English 
% citation first, followed by the original foreign-language citation \cite{b6}.

\begin{thebibliography}{00}
\bibitem{bben1} Weir, C.S., Douglas, G., Carruthers, M. and Jack, M., 2009. User perceptions of security, convenience and usability for ebanking authentication tokens. Computers \& Security, 28(1-2), abstract.
\end{thebibliography}
\end{document}
